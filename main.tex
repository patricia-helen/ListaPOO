\documentclass{article}

\usepackage[portuguese]{babel}
%\usepackage[a4paper,top=2cm,bottom=2cm,left=3cm,right=3cm,marginparwidth=1.75cm]{geometry}
\usepackage[a4paper,top=2cm,bottom=2cm,left=2.5cm,right=2.5cm]{geometry}

\usepackage{amssymb}
\usepackage{siunitx}
\PassOptionsToPackage{hyphens}{url}\usepackage{hyperref}
\usepackage{cleveref}
\usepackage[utf8]{inputenc}
\usepackage{csquotes}
\usepackage{booktabs}
\usepackage{longtable}
\usepackage{adjustbox}
\usepackage{array}
\usepackage{url}
\usepackage{titlesec}
\usepackage{authblk}
\usepackage{xcolor}
\renewcommand{\thetable}{\Roman{table}}
\usepackage{graphicx} % Para usar \scalebox
\usepackage{pgf-umlsd}
\usepackage{tikz-uml}

% Definir formato para \subsection
\titleformat{\subsection}{\mdseries\itshape\large}{\thesubsection}{1em}{} 

\title{Modelagem de Sistema para Gestão de Academia em Milho Verde}
\author[1]{Patricia Helen Ribeiro Moreira}
\affil[1]{Universidade Federal dos Vales do Jequitinhonha e Mucuri (UFVJM), Campus Diamantina, MG, Brasil\\\texttt{patricia.helen@ufvjm.edu.br}}

\begin{document}
\maketitle

\begin{abstract}
\textbf{Objetivo} - Este trabalho visa desenvolver um modelo de sistema para gerenciar uma academia localizada em Milho Verde, abordando desde o controle de agendamentos até a gestão financeira completa da academia.

\textbf{Metodologia/Abordagem} - O sistema foi modelado utilizando conceitos de Programação Orientada a Objetos (POO) e Engenharia de Software. A modelagem inclui diagramas de casos de uso, sequência, classe e estados.

\textbf{Palavras-chave:} Gerenciamento de Academia; Sistema de Agendamento; Programação Orientada a Objetos; Engenharia de Software; Gestão Financeira.
\end{abstract}

\section{Introdução}
A crescente demanda por serviços de academias, especialmente em áreas turísticas como Milho Verde, exige um sistema robusto e eficiente para gerenciar tanto as operações diárias quanto as necessidades administrativas. Este trabalho tem como objetivo principal o desenvolvimento de um modelo de sistema para gerenciar uma academia em Milho Verde. O sistema visa otimizar o processo de agendamento de aulas, o lançamento de despesas diárias e a geração de balanços mensais.

Utilizando conceitos de Programação Orientada a Objetos (POO) e práticas recomendadas de Engenharia de Software, o sistema foi modelado para atender às especificidades de uma academia que recebe tanto clientes locais quanto turistas. A modelagem inclui a elaboração de diagramas de casos de uso, sequência, classe e estados, fornecendo uma visão completa do funcionamento do sistema proposto. A implementação futura deste sistema pode facilitar o gerenciamento de academias em contextos semelhantes, proporcionando uma ferramenta eficaz para a administração e operação.

\section{Diagrama de Caso de Uso}
\begin{tikzpicture}
    \begin{umlsystem}[x=0, y=0]{Sistema Academia}
        \umlusecase[x=1, y=10, name=VerificarVaga]{Verificar vaga na agenda}
        \umlusecase[x=9, y=11, name=Agendar]{Realizar agendamento}
        \umlusecase[x=9, y=7, name=Confirmar]{Confirmar agendamento}
        \umlusecase[x=3, y=8, name=Cancelar]{Cancelar agendamento}
        \umlusecase[x=0, y=7, name=LancarDespesa]{Lançar despesa diária}
        \umlusecase[x=1, y=2, name=GerarBalanco]{Gerar balanço mensal}
        \umlusecase[x=2, y=4, name=GerenciarDespesas]{Gerenciar despesas da academia}
        \umlusecase[x=1, y=12, name=Autenticar]{Autenticar Usuário}
    \end{umlsystem}

    \umlactor[x=-4, y=8]{Funcionario}
    \umlactor[x=-4, y=4]{Proprietario}
    
    % Generalização entre Proprietário e Funcionário com seta vazia
   \umlinherit{Proprietario}{Funcionario}
    
    % Associações do Funcionário
    \umlassoc{Funcionario}{Autenticar}
    \umlassoc{Funcionario}{VerificarVaga}
    \umlassoc{Funcionario}{Agendar}
    \umlassoc{Funcionario}{LancarDespesa}
    
    % Associações do Proprietário
    \umlassoc{Proprietario}{GerarBalanco}
    \umlassoc{Proprietario}{GerenciarDespesas}
    
    % Relacionamentos de dependência (include/extend)
    \umlinclude{Agendar}{VerificarVaga} % Inclusão de "Verificar Vaga" no "Agendar"
    \umlinclude{Agendar}{Confirmar}
    \umlinclude{Agendar}{Cancelar}

\end{tikzpicture}


\section{Fluxo de Eventos}
\section*{Verificar vaga na agenda}

O Funcionário/Proprietário deve seguir os seguintes passos:
\begin{itemize}
    \item Autenticar-se no sistema utilizando seu login e senha.
    \item Escolher a opção "Verificar vaga na agenda".
    \item Inserir a data específica para a qual deseja verificar as vagas disponíveis.
    \item Selecionar a aula desejada para verificar os horários e o número de vagas disponíveis.
    \item Visualizar os detalhes das vagas disponíveis, incluindo os horários e os clientes já agendados.
    \item Caso deseje, pode proceder com o "Realizar agendamento" para marcar uma vaga.
    \item Se não desejar realizar outra ação, encerrar a consulta no sistema.
\end{itemize}

\newpage % Quebra de página antes do próximo cenário

\section*{Realizar um agendamento preliminar}

O Funcionário deve seguir os seguintes passos:
\begin{itemize}
    \item Autenticar-se no sistema utilizando seu login e senha (caso de uso: \textit{Autenticar Usuário}).
    \item Escolher a opção "Realizar agendamento preliminar" (caso de uso: \textit{Realizar agendamento}).
    \item Inserir as informações do cliente, incluindo nome, endereço, número de telefone, e-mail e CPF pseudo-anonimizado.
    \item Selecionar a data e a aula desejada (spinning, musculação, fit dance ou pilates).
    \item O sistema verifica automaticamente a disponibilidade de vagas na agenda para a data e aula selecionadas (caso de uso: \textit{Verificar vaga na agenda}).
    \item Caso haja vagas disponíveis, o funcionário insere as informações adicionais, como o número do cartão de crédito do cliente, o tipo de aula, e o instrutor responsável.
    \item O sistema registra o agendamento preliminar e exibe um resumo da reserva feita, incluindo o preço a ser cobrado.
    \item O funcionário confirma o agendamento preliminar ou decide encerrar a operação.
    \item O sistema salva o agendamento preliminar e aguarda a confirmação do cliente até 5 dias úteis antes da data marcada.
\end{itemize}

\section*{Cancelar um agendamento}

O sistema deve seguir os seguintes passos:
\begin{itemize}
    \item Verificar automaticamente se o agendamento preliminar não foi confirmado pelo cliente até 5 dias úteis antes da data marcada.
    \item Se o agendamento não for confirmado, o sistema automaticamente altera o status do agendamento para "Cancelado".
    \item O sistema atualiza a disponibilidade da vaga na agenda, liberando o horário para outros clientes.
    \item O sistema notifica o cliente sobre o cancelamento automático do agendamento via e-mail ou SMS, conforme as informações de contato fornecidas.
    \item Se o cancelamento ocorrer após a confirmação do agendamento, o sistema processa automaticamente o reembolso de 50\% do valor cobrado, conforme as regras estabelecidas.
    \item O sistema registra o cancelamento e finaliza a operação.
\end{itemize}

\section*{Lançar despesa diária de um Cliente/Academia}

O Funcionário/Proprietário deve seguir os seguintes passos:
\begin{itemize}
    \item Autenticar-se no sistema utilizando seu login e senha.
    \item Escolher a opção "Lançar despesa diária".
    \item Selecionar o tipo de despesa a ser lançada:
    \begin{itemize}
        \item Despesa do cliente: relacionada ao consumo em lanchonete, compra de material na loja, etc.
        \item Despesa da academia: relacionada à operação geral da academia, como limpeza, materiais de uso diário, pagamento de instrutores, etc.
    \end{itemize}
    \item Se for uma **despesa do cliente**:
    \begin{itemize}
        \item Selecionar o espaço onde a despesa foi realizada (ex: lanchonete, loja de material de academia).
        \item O sistema exibe o estoque de produtos disponíveis no espaço selecionado, incluindo quantidades e preços.
        \item O funcionário seleciona os produtos/serviços que o cliente consumiu ou adquiriu, adicionando-os à comanda do cliente.
        \item O sistema atualiza automaticamente o estoque, adiciona a data e hora da transação, e calcula o valor total da comanda.
        \item O funcionário oferece as opções de pagamento ao cliente: cartão, dinheiro ou PIX.
        \item O funcionário seleciona a forma de pagamento utilizada pelo cliente e confirma o pagamento.
        \item O sistema registra a despesa no histórico financeiro do cliente e atualiza o saldo devedor.
        \item O sistema gera um comprovante digital da comanda, que pode ser enviado ao cliente e adicionado ao histórico de transações.
    \end{itemize}
    \item Se for uma **despesa da academia**:
    \begin{itemize}
        \item Selecionar a categoria de despesa:
        \begin{itemize}
            \item Limpeza;
            \item Material de uso diário;
            \item Pagamento de instrutores (acesso exclusivo do Proprietário).
        \end{itemize}
        \item Para as despesas de **limpeza** ou **material de uso diário**:
        \begin{itemize}
            \item O sistema exibe o estoque de materiais disponíveis para a categoria selecionada.
            \item O funcionário seleciona os materiais utilizados, incluindo a quantidade.
            \item Se o item necessário não estiver no estoque, o sistema oferece a opção "Criar nova despesa".
            \item Caso a opção "Criar nova despesa" seja selecionada:
            \begin{itemize}
                \item O funcionário insere a descrição, o nome, e o valor da nova despesa.
                \item O sistema adiciona automaticamente a nova despesa ao estoque ou à lista de serviços disponíveis, para uso futuro.
                \item A nova despesa é então selecionada e adicionada à transação atual.
            \end{itemize}
            \item O sistema atualiza automaticamente o estoque, adiciona a data e hora da transação, e calcula o valor total da despesa.
            \item O sistema registra a despesa no histórico financeiro da academia e atualiza o saldo de despesas do mês.
            \item O sistema gera um relatório interno da despesa, que pode ser acessado posteriormente para fins de balanço mensal.
        \end{itemize}
        \item Para o **pagamento de instrutores**:
        \begin{itemize}
            \item O proprietário acessa a opção "Pagamento de instrutores".
            \item O proprietário insere as informações de pagamento, como o valor (a data e hora são preenchidas automaticamente pelo sistema).
            \item O sistema registra o pagamento no histórico financeiro da academia e atualiza o saldo de despesas do mês.
            \item O sistema gera um relatório interno da despesa, que pode ser acessado pelo proprietário para fins de balanço mensal.
        \end{itemize}
    \end{itemize}
    \item **Editar Despesa (acesso exclusivo do Proprietário):**
    \begin{itemize}
        \item O proprietário seleciona a opção "Editar Despesa".
        \item O proprietário navega até a despesa que deseja editar, seja uma despesa do cliente ou da academia.
        \item O sistema permite a edição dos detalhes da despesa, como descrição, valor, e quantidade.
        \item O sistema atualiza os registros da despesa editada e ajusta o estoque e/ou o histórico financeiro conforme necessário.
        \item O sistema gera um novo relatório, refletindo as mudanças feitas, e arquiva as alterações para auditoria futura.
    \end{itemize}
    \item O funcionário ou proprietário encerra a operação ou inicia o processo para registrar ou editar uma nova despesa.
\end{itemize}

\section*{Gerar balanço mensal}

O Proprietário deve seguir os seguintes passos:
\begin{itemize}
    \item Autenticar-se no sistema utilizando seu login e senha (caso de uso: \textit{Autenticar Usuário}).
    \item Escolher a opção "Gerar balanço mensal" no menu principal.
    \item O sistema verifica automaticamente se o mês atual já foi concluído:
    \begin{itemize}
        \item Se o mês ainda não terminou (menos de 29 dias no mês), o sistema exibe uma mensagem informando que o balanço mensal só poderá ser gerado após o término do mês.
        \item Se o mês terminou (mínimo de 29 dias, máximo de 31 dias), o sistema permite a geração do balanço mensal.
    \end{itemize}
    \item O sistema solicita a seleção do mês e ano para o qual o balanço deve ser gerado:
    \begin{itemize}
        \item Se o mês ainda não terminou, a opção de geração do balanço não está disponível.
        \item Se o mês terminou, o proprietário seleciona o mês e o ano desejado.
    \end{itemize}
    \item O sistema compila todas as receitas e despesas do período selecionado, incluindo:
    \begin{itemize}
        \item Receitas provenientes de diárias, mensalidades e vendas de produtos.
        \item Despesas relacionadas a salários de instrutores, limpeza, materiais de uso diário, e quaisquer outras despesas lançadas no sistema durante o mês.
    \end{itemize}
    \item O sistema calcula o saldo final do mês, considerando todas as receitas e despesas.
    \item O sistema gera um relatório detalhado do balanço mensal, que inclui:
    \begin{itemize}
        \item Resumo das receitas e despesas;
        \item Gráficos e tabelas que mostram a distribuição das despesas por categoria;
        \item Saldo final do mês;
        \item Comparação com meses anteriores (se aplicável).
    \end{itemize}
    \item O relatório é salvo automaticamente no histórico financeiro da academia.
    \item O proprietário pode optar por:
    \begin{itemize}
        \item Consultar relatórios de balanços mensais anteriores, selecionando o mês e ano desejado.
        \item Salvar o relatório gerado em formato digital para consulta futura;
        \item Imprimir o relatório;
        \item Enviar o relatório por e-mail para stakeholders ou para o próprio arquivo.
    \end{itemize}
    \item O proprietário encerra a operação ou opta por consultar outro relatório.
\end{itemize}



\section{Diagrama de Sequência}
\section*{Verificar Vaga na Agenda}
\begin{sequencediagram}
  \newthread{func}{Funcionário}
  \newinst[3]{sys}{Sistema}
  \newinst[3]{agenda}{Agenda}

  \begin{call}{func}{Autenticar()}{sys}{}
  \end{call}

  \begin{call}{func}{VerificarVaga(data, aula)}{sys}{}
    \begin{call}{sys}{ConsultarVagas(data, aula)}{agenda}{vagasDisponíveis}
    \end{call}
  \end{call}

  \begin{call}{sys}{ExibirVagas(vagasDisponíveis)}{func}{}
  \end{call}
\end{sequencediagram}

\vspace{1cm}

\section*{Realizar Agendamento Preliminar}
\begin{sequencediagram}
  \newthread{func}{Funcionário}
  \newinst[3]{sys}{Sistema}
  \newinst[3]{agenda}{Agenda}
  \newinst[3]{cli}{Cliente}

  \begin{call}{func}{Autenticar()}{sys}{}
  \end{call}

  \begin{call}{func}{AgendarPreliminar(cliInfo, aula, data)}{sys}{}
    \begin{call}{sys}{VerificarVaga(data, aula)}{agenda}{vagaDisponível}
    \end{call}
    \begin{call}{sys}{RegistrarAgendamento(cliInfo, aula, data)}{agenda}{resumoAgendamento}
    \end{call}
  \end{call}

  \begin{call}{sys}{ExibirResumo(resumoAgendamento)}{func}{}
  \end{call}

  \begin{call}{func}{ConfirmarAgendamento()}{sys}{}
  \end{call}
\end{sequencediagram}

\vspace{1cm}

\section*{Cancelar um Agendamento}
\begin{sequencediagram}
  \newthread{func}{Funcionário}
  \newinst[3]{sys}{Sistema}
  \newinst[3]{agenda}{Agenda}

  \begin{call}{func}{Autenticar()}{sys}{}
  \end{call}

  \begin{call}{func}{CancelarAgendamento(agendamentoId)}{sys}{}
    \begin{call}{sys}{RemoverAgendamento(agendamentoId)}{agenda}{}
    \end{call}
  \end{call}
\end{sequencediagram}

\vspace{1cm}

\section*{Lançar Despesa Diária de um Cliente/Academia}
\begin{sequencediagram}
  \newthread{func}{Funcionário}
  \newinst[3]{sys}{Sistema}
  \newinst[3]{estoque}{Estoque}

  \begin{call}{func}{Autenticar()}{sys}{}
  \end{call}

  \begin{call}{func}{SelecionarEspaço(espaço)}{sys}{}
  \end{call}

  \begin{call}{sys}{ListarProdutosServiços(estoque)}{estoque}{itensDisponíveis}
  \end{call}

  \begin{call}{func}{SelecionarItens(itens)}{sys}{}
  \end{call}

  \begin{call}{sys}{RegistrarDespesa(cliente, itens)}{estoque}{}
  \end{call}

  \begin{call}{sys}{ExibirResumoDespesa(resumo)}{func}{}
  \end{call}

  \begin{call}{func}{ProcessarPagamento(métodoPagamento)}{sys}{}
  \end{call}
\end{sequencediagram}

\vspace{1cm}

\section*{Gerar Balanço Mensal}
\begin{sequencediagram}
  \newthread{prop}{Proprietário}
  \newinst[3]{sys}{Sistema}
  \newinst[3]{relatorio}{RelatórioMensal}

  \begin{call}{prop}{Autenticar()}{sys}{}
  \end{call}

  \begin{call}{prop}{GerarBalançoMensal(mês)}{sys}{}
    \begin{call}{sys}{CompilarDespesasReceitas(mês)}{relatorio}{resumoMensal}
    \end{call}
  \end{call}

  \begin{call}{sys}{SalvarRelatório(resumoMensal)}{relatorio}{}
  \end{call}

  \begin{call}{sys}{ExibirResumoMensal(resumoMensal)}{prop}{}
  \end{call}
\end{sequencediagram}

\section{Diagrama de Classes}
\clearpage
\begin{center} % Centralizar o diagrama
\scalebox{0.7}{ % Reduzir a escala para 70%

\begin{tikzpicture}
    \umlclass[x=-6, y=0]{Pessoa}{
        - id: int \\
        - nome: string \\
        - login: string \\
        - senha: string
    }{
        + autenticar(): boolean
    }

    \umlclass[x=4, y=0]{Funcionario}{
        - cargo: string
    }{
        + verificarVaga() \\
        + realizarAgendamentoPreliminar() \\
        + cancelarAgendamento() \\
        + lancarDespesa()
    }
    
    \umlclass[x=-6, y=-7]{Proprietario}{
    }{
        + gerarBalancoMensal() \\
        + editarDespesa()
    }
    
    \umlclass[x=-4, y=-4]{Cliente}{
        - id: int \\
        - nome: string \\
        - email: string \\
        - telefone: string
    }{
    }

    \umlclass[x=1, y=-4]{Agendamento}{
        - id: int \\
        - data: date \\
        - hora: time \\
        - status: string \\
        - tipo: string \\
        - clienteId: int
    }{
        + confirmarAgendamento() \\
        + cancelarAgendamento()
    }
    
    \umlclass[x=6, y=-4]{Despesa}{
        - id: int \\
        - descricao: string \\
        - valor: float \\
        - data: date \\
        - tipo: string \\
        - clienteId: int \\
        - espacoId: int
    }{
        + criarNovaDespesa()
    }
    
    \umlclass[x=4, y=-10]{Espaco}{
        - id: int \\
        - nome: string
    }{
        + listarProdutosServicos()
    }
    
    \umlclass[x=-4, y=-10]{Produto}{
        - id: int \\
        - nome: string \\
        - valor: float \\
        - quantidade: int \\
        - espacoId: int
    }{
    }
    
    \umlclass[x=0, y=-15]{Estoque}{
        - espacoId: int \\
        - produtoId: int \\
        - quantidadeDisponivel: int
    }{
    }
    
    \umlclass[x=-6, y=-15]{RelatorioMensal}{
        - mes: int \\
        - ano: int \\
        - despesasTotais: float \\
        - receitasTotais: float \\
        - balanco: float
    }{
        + compilarDespesasReceitas() \\
        + salvarRelatorio() \\
        + exibirResumoMensal()
    }
    
    % Herança
    \umlinherit{Funcionario}{Pessoa}
    \umlinherit{Proprietario}{Pessoa}
    
    % Associação
    \umlassoc{Funcionario}{Agendamento}
    \umlassoc{Funcionario}{Despesa}
    \umlassoc{Proprietario}{RelatorioMensal}
    \umlassoc{Espaco}{Produto}
    \umlassoc{Produto}{Estoque}
    \umlassoc{Cliente}{Agendamento}
    \umlassoc{Despesa}{Espaco}
\end{tikzpicture}
} % Fim da escala
\end{center} % Fim do centro
\clearpage


\section{Diagrama de Estado}
\usetikzlibrary{automata, positioning}

\begin{tikzpicture}[shorten >=1pt, node distance=3cm and 4cm, on grid, auto]
   % Definição dos estados
   \node[state, initial] (iniciada)   {Iniciada};
   \node[state] (confirmada) [right=of iniciada] {Confirmada};
   \node[state] (paga) [below right=of confirmada] {Paga};
   \node[state] (cancelada) [below=of confirmada] {Cancelada};
   \node[state, accepting] (concluida) [right=of paga] {Concluída};

   % Definição das transições
   \path[->]
    (iniciada) edge [bend left] node {Criar Reserva} (confirmada)
               edge [bend right] node [below left] {Cancelar Reserva} (cancelada)
    (confirmada) edge [bend left=15] node [above] {Confirmar Pagamento} (paga)
                 edge [bend right] node [below] {Cancelar Reserva} (cancelada)
    (paga) edge [bend left=20] node [below right] {Utilizar Reserva} (concluida);
\end{tikzpicture}

\newpage % Quebra de página antes da conclusão
\section{Conclusão}
Neste trabalho, foi desenvolvido um modelo de sistema para gestão de academias em Milho Verde, utilizando conceitos de Programação Orientada a Objetos (POO) e práticas de Engenharia de Software. A modelagem incluiu diagramas de casos de uso, sequência, classes e estados, proporcionando uma visão abrangente e detalhada do funcionamento do sistema proposto.

O sistema modelado busca atender às necessidades operacionais e administrativas de uma academia localizada em uma região turística, oferecendo funcionalidades que vão desde o controle de agendamentos até a gestão financeira completa. A abordagem adotada permite uma maior eficiência no gerenciamento das atividades diárias, garantindo um melhor atendimento aos clientes, tanto locais quanto turistas.

A aplicação dos conceitos de POO na modelagem do sistema demonstrou ser uma escolha eficaz, permitindo uma estrutura modular e extensível, que pode ser facilmente adaptada a outras academias com características similares. A utilização de diagramas detalhados auxiliou na compreensão dos processos e na identificação de possíveis melhorias e adaptações futuras.

Este estudo, além de oferecer uma solução prática para o gerenciamento de academias, contribui para a literatura de Engenharia de Software ao demonstrar como os conceitos de POO podem ser aplicados de forma eficiente no desenvolvimento de sistemas de gestão. Futuras implementações deste modelo podem contribuir para uma maior eficiência operacional e uma melhor experiência para os usuários das academias, beneficiando tanto os administradores quanto os clientes.

\hfill
\end{document}
